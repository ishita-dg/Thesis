%!TEX root = ../dissertation.tex
% the abstract

%should state the problem, describe the methods and procedures used, and give the main results or conclusions of the research.

How do humans reason intelligently in a complex and ever-changing world within limits on energy, data and time? How might an understanding of this help us build human-like artificial intelligence? Structured Bayesian models provide a normative account of rational behavior. Although computing these rational responses via exact Bayesian inference is often computationally intractable, empirical findings show that human behavior is often consistent with these rational responses. In some cases, however, humans display `cognitive biases' such that their responses deviate systematically from the normative Bayesian response. How can we reconcile these? This thesis makes progress toward this reconciliation by building on the insight that humans are not general purpose computing machines---we are instead `ecologically rational', adapting to structure in our environments to make the best use of our limited resources. Chapters \ref{chap:approx} \& \ref{chap:amort} discuss algorithms for approximating exact Bayesian inference within limitations on computational resources. These reduce the computational costs of inference by leveraging underlying environmental structure through a process of \textit{amortization} (the intelligent re-use of previous computations). However, amortization can lead to errors when the current query is not representative of past experience. Chapters \ref{chap:MCMC}--\ref{chap:LTI} demonstrate that these errors replicate several human cognitive biases. New predictions are tested in several behavioral experiments. Chapters \ref{chap:sentences} \& \ref{chap:causal} demonstrate that amortized computations can give rise to ecologically rational behaviors in machine learning, and show how this can be leveraged to artificially engineer new kinds of intelligent behaviors like causal reasoning and compositional language understanding. This also provides new insights into how these central tenets of intelligence arise in humans. By taking an algorithmic approach to ecological rationality---i.e. making explicit claims about how it can be implemented at the level of computational processes---this thesis develop new models for human probabilistic inference that can explain both its remarkable successes and seeming failures, as well as suggest new paths toward machines with human-like intelligence.