%!TEX root = ../dissertation.tex
\chapter{Conclusion}
\label{chap:conclusion}

In 1955, Herb Simon put forth the challenge facing more realistic theories of human intelligence: "Broadly stated, the task is to replace the global rationality of economic man with a kind of rational behavior that is compatible with the access to information and the computational capacities that are actually possessed by organisms, including man, in the kinds of environments in which such organisms exist." This thesis hopes to do exactly that. By taking into account the circumstances under which intelligent behavior manifests -- both the limitations on resources, structure in the environment, and how these two interact -- we provide new computational models of human probabilistic inference, that are psychologically plausible. Without plausible algorithmic solutions to rational or normative inference in structured Bayesian models, they remain unsatisfying as models of human cognition, and we cannot leverage their many desirable properties in building intelligent machines. The ideas furthered in this thesis, of leveraging environmental structure via flexible re-use of previous computations to simply inference, bring such Bayesian models of intelligent behavior back into the realm of the possible. 

Further, these models parsimoniously explain a wide range of empirical findings about non-normative inference, and how humans can sometimes be so close to optimal, and in other domains (with the same cognitive resources), so biased -- and biased in so many different context-sensitive ways. These insights also lead to entirely new ways to understand and engineer artificial systems, via manipulation of the environments in which they learn and function. This confluence suggests links between the analysis of ecological rationality in humans and in machines, leading to new lines of research into understanding both.

\section*{Open Questions and Future Work}

The key question left open is how the probabilistic models are acquired in the first place. Do humans meta-learn?

question of representation.

flexible re-use, compostional re-use.

continuum of model and inference.
