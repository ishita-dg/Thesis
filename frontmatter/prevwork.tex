%!TEX root = ../dissertation.tex

{\setlength{\parindent}{0em}
The research presented in this thesis is based on a selection of papers that I have worked on during my PhD.\\

Chapter \ref{chap:MCMC} is based on: Dasgupta, I., Schulz, E., \& Gershman, S. J. (2017). Where do hypotheses come from?. \textit{Cognitive psychology}, 96, 1-25.\\


Chapter \ref{chap:MCMC_amort} is based on:  Dasgupta, I., Schulz, E., Goodman, N., \& Gershman, S. (2017). Amortized hypothesis generation.  In \textit{Proceedings of the annual meeting of the cognitive science society.} (Vol. 39, No. 39).\\ Dasgupta, I., Schulz, E., Goodman, N. D., \& Gershman, S. J. (2018). Remembrance of inferences past: Amortization in human hypothesis generation. \textit{Cognition}, 178, 67-81.\\


Chapter \ref{chap:LTI} is based on: Dasgupta, I., Schulz, E., Tenenbaum, J. B., \& Gershman, S. J. (2019). A theory of learning to infer.  \textit{Psychological Review} (in press).\\


Chapter \ref{chap:sentences} is based on: Dasgupta, I., Guo, D., Stuhlmüller, A., Gershman, S. J., \& Goodman, N. D. (2018). Evaluating compositionality in sentence embeddings. In \textit{Proceedings of the annual meeting of the cognitive science society.} (Vol. 40, No. 40). \\ Dasgupta, I., Guo, D., Gershman, S. J., \& Goodman, N. D. (2019). Analyzing machine-learned representations: A natural language case study. \textit{arXiv preprint arXiv:1909.05885} (under review).\\


Chapter \ref{chap:causal} is based on: Dasgupta, I., Wang, J., Chiappa, S., Mitrovic, J., Ortega, P., Raposo, D., ... \& Kurth-Nelson, Z. (2019). Causal reasoning from meta-reinforcement learning. \textit{arXiv preprint arXiv:1901.08162}. \\ Dasgupta, I., Kurth-Nelson, Z., Chiappa, S., Mitrovic, J., Hughes, E., Botvinick, M., \& Wang, J. (2019). Meta-reinforcement learning of causal strategies.  In \textit{Advances in neural information processing systems} (MetaLearn 2019 workshop).\\


Other related works not included in this thesis: Bernstein, J.*, Dasgupta, I.*, Rolnick, D.*, \& Sompolinsky, H. (2017). Markov transitions between attractor states in a recurrent neural network. In \textit{AAAI Spring Symposium Series}. \\ Dasgupta, I.*, Smith, K. A.*, Schulz, E., Tenenbaum, J. B., \& Gershman, S. J. (2018). Learning to act by integrating mental simulations and physical experiments.  In \textit{Proceedings of the annual meeting of the cognitive science society.} (Vol. 40, No. 40).\\
* denotes equal contribution.
}